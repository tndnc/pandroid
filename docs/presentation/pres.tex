%--------------------------------------------------
%	DOCUMENT IMPORTS
%--------------------------------------------------
\documentclass{beamer}
\usetheme{CambridgeUS}
\usepackage[utf8]{inputenc}
\usepackage[T1]{fontenc}
\usepackage[french]{babel}
\usepackage{graphicx}
\usepackage{xcolor}
\usepackage{tikz}
\usepackage{tkz-graph}
\usepackage{amsmath, amssymb, mathtools, amsthm}
%-----------------------------------------------
% DOCUMENT CONFIG
%-----------------------------------------------
\graphicspath{ {figures/} }
\useinnertheme{rectangles}
\setbeamertemplate{blocks}[default]
\usefonttheme[onlymath]{serif}
% Tikz
\tikzstyle{vertex}=[circle, draw, inner sep=2pt, minimum size=8pt, thick]
\newcommand{\vertex}{\node[vertex]}
\usetikzlibrary{arrows,petri,topaths,calc}
% Commands
\newcommand{\yy}[1]{\colorbox{yellow!70}{#1}}
\newcommand{\bb}[1]{\colorbox{blue!30}{#1}}
\newcommand{\rr}[1]{\colorbox{red!30}{#1}}
\newcommand{\gr}[1]{\colorbox{green!30}{#1}}

\title[Pas de jaloux]{Pas de jaloux, un jeu de partage équitable}
\author[Bontems, Mottola, Thirunavukarasu]{Alexandre Bontems \and Gualtiero Mottola \and Hans Thirunavukarasu}
\institute[]{Faculté des Sciences, Sorbonne Université\\Université Pierre et Marie Curie}
\date{5 juin 2018}

\setcounter{tocdepth}{1}
\AtBeginSection[]
{
    \begin{frame}
        \frametitle{Table des matières}
        \tableofcontents[currentsection]
    \end{frame}
}

\begin{document}
\frame{\maketitle}
\section{Introduction}
\begin{frame}
    \frametitle{Présentation du problème LEF}
	\begin{center}
	Problème de satisfaction NP-complet.\\[1cm]
	    \begin{tikzpicture}
    	    % Agents
	        \vertex (a1) at (0,0) {$a_1$};
	        \vertex (a2) at (1.5,0) {$a_2$};
	        \vertex (a3) at (3,0) {$a_3$};
	        \vertex (a4) at (4.5,0) {$a_4$};
	        % Prefs
	        \node (a1o1) at (0,2.2) {$o_2$};
	        \node (a1o2) at (0,1.7) {$o_4$};
	        \node (a1o3) at (0,1.2) {\yy{$o_1$}};
	        \node (a1o4) at (0,0.7) {$o_3$};
	        \node (a2o1) at (1.5,2.2) {\yy{$o_3$}};
	        \node (a2o2) at (1.5,1.7) {$o_4$};
	        \node (a2o3) at (1.5,1.2) {$o_2$};
	        \node (a2o4) at (1.5,0.7) {$o_1$};
	        \node (a3o1) at (3.0,2.2) {\yy{$o_2$}};
	        \node (a3o2) at (3.0,1.7) {$o_1$};
	        \node (a3o3) at (3.0,1.2) {$o_4$};
	        \node (a3o4) at (3.0,0.7) {$o_3$};
	        \node (a4o1) at (4.5,2.2) {$o_3$};
	        \node (a4o2) at (4.5,1.7) {\yy{$o_4$}};
	        \node (a4o3) at (4.5,1.2) {$o_2$};
	        \node (a4o4) at (4.5,0.7) {$o_1$};
	        \path
	        (a1) edge[thick] (a2)
            (a2) edge[thick] (a3)
            (a3) edge[thick] (a4)       
	        ;
	    \end{tikzpicture}
	\end{center}
\end{frame}

\begin{frame}
\frametitle{Objectifs}
Développement d'un jeu puzzle basé sur le problème et offrant une progression de difficulté.

\begin{block}{Solutions}
\begin{itemize}
    \item Application Android
    \item Outils d'analyse de difficulté
    \item Taille des instances bornée
\end{itemize}
\end{block}
\end{frame}

\section{Application}
\subsection{Conception de l'interface}
\begin{frame}
\frametitle{Conception de l'interface}
\begin{itemize}
    \item Processus itératif
    \item Public visé large
    \item Ne pas introduire de biais 
\end{itemize}
\end{frame}
\begin{frame}
\frametitle{Interface de jeu}
\end{frame}

\begin{frame}
\frametitle{Tutoriel}
\end{frame}

\subsection{Architecture}
\begin{frame}
\frametitle{Diagramme de classe ?}
\end{frame}

\subsection{Récupération des données}
\begin{frame}
\frametitle{Intégration GoogleSheets}
\end{frame}

\section{Analyse}
\begin{frame}
\frametitle{Génération d'instances solvables}
\end{frame}

\begin{frame}
\frametitle{Résolution par backtrack}
\end{frame}

\begin{frame}
\frametitle{Résolution ASP}
\end{frame}

\begin{frame}
\frametitle{Analyse de fitness landscape}
\end{frame}

\begin{frame}
\frametitle{Apprentissage}
\end{frame}

\section{Conclusion}
\begin{frame}
\frametitle{Problèmes rencontrés}
\end{frame}

\begin{frame}
\frametitle{Travail futur}
\end{frame}
\end{document}