%--------------------------------------------------
%	DOCUMENT CONFIGURATION
%--------------------------------------------------
\documentclass[a4paper, 11pt, titlepage]{article}
\usepackage[utf8]{inputenc}
\usepackage[T1]{fontenc}
\usepackage[french]{babel}
\usepackage[margin=1in]{geometry}
\usepackage{amsmath, amssymb, mathtools, amsthm}
\usepackage{graphicx}
\usepackage{subfiles}
\usepackage[table]{xcolor}
\usepackage{titlesec}
\usepackage{tikz}
\usepackage{tkz-graph}
\usepackage{listings}
\usepackage{subcaption}
\usepackage[backend=biber,style=numeric,citestyle=numeric]{biblatex}
\usepackage{csquotes}
\usepackage{changepage}
\usepackage{appendix}
\usepackage{hyperref}
\usepackage{cleveref}
\usepackage{tikz-uml}
%-----------------------------------------------
% DOCUMENT CONFIG
%-----------------------------------------------
% Add point after title number
\titleformat{\section}[block]{\sc\bfseries\center\Large}{\thesection.}{0.5em}{}
\titleformat{\subsection}[block]{\sc\bfseries\center}{\thesubsection.}{0.5em}{}
\titleformat{\subsubsection}[block]{\sc\bfseries\center}{\thesubsubsection.}{0.5em}{}
% Commands
\newcommand{\yy}[1]{\colorbox{yellow!70}{#1}}
\newcommand{\bb}[1]{\colorbox{blue!30}{#1}}
\newcommand{\rr}[1]{\colorbox{red!30}{#1}}
\newcommand{\gr}[1]{\colorbox{green!30}{#1}}
% Tikz
\tikzstyle{vertex}=[circle, draw, inner sep=2pt, minimum size=8pt]
\newcommand{\vertex}{\node[vertex]}
\usetikzlibrary{arrows,petri,topaths,calc}
% DOC INFO
\title{Rapport de Projet Tuteuré - ANDROIDE}
\author{Alexandre Bontems, Gualtiero Mottola, Hans Thirunavukarasu}
% Listings
\lstset{
frame=lines,
basicstyle=\ttfamily\small,
numbers=left,
numberstyle=\tiny,
%numbersep=5pt,
literate=%
{é}{{\'e}}1
{ê}{{\^e}}1
{à}{{\`a}}1,
commentstyle=\color{gray},
language=Octave
}
% Theorems
\newtheorem{observation}{\it\bfseries Observation}
\newtheorem{definition}{\it\bfseries Définition}
%images storage 
\graphicspath{ {./images/} {./sections/images/} }
% Bibliography
\addbibresource{bibliography.bib}
%--------------------------------------------------
%	DOCUMENT BODY
%--------------------------------------------------
\begin{document}	
	\begin{center}
	    \includegraphics[width=0.3\linewidth]{sorbonne}\\[0.8cm]
	    \textsf{\large\bfseries Rapport de Projet Tuteuré - PANDROIDE}\\[0.25cm]
	    \textsf{\Large\bfseries Pas de jaloux, un jeu de partage équitable}\\[0.5cm]
	    \textbf{Auteurs}\\
	    Alexandre Bontems\\Gualtiero Mottola\\Hans Thirunavukarasu\\[0.25cm]
	    \textbf{Superviseurs}\\
	    Nicolas Maudet\\Aurélie Beynier\\[0.5cm]
	    \textit{Université Pierre et Marie Curie, Paris 6, Département Informatique\\4 place Jussieu 75252 Paris cedex 05, France}\\[0.5cm]
	\end{center}
	\begin{adjustwidth}{.5in}{.5in}\small
	    Dans le cadre du master informatique \textsf{ANDROIDE} de l'UPMC, un projet tuteuré doit être effectué par les étudiants et ce rapport en détaille les résultats. Le problème de partage équitable LEF (Local Envy Freeness), présenté en \cite{lef}, est étudié et un jeu puzzle en est dérivé. Pour répondre aux problématiques d'analyse de difficulté pour l'humain, des résolutions \textquote{à la main} sont observées et des outils sont développés pour tenter d'expliquer les ressentis. Le développement d'une application jeu pose également les problématiques liées à l'expérience utilisateur et de conception de tutoriel. 
	\end{adjustwidth}
	\tableofcontents
%	\newpage

	\section{Introduction}
		\subfile{sections/intro.tex}
	\section{Analyse d'instances}
	    \label{sec-analyse}
        \subfile{sections/analyse.tex}
	\section{Application mobile}
	    \label{app}
	    \subfile{sections/app.tex}
	    
	\section{Travail futur}
	
	
	\section{Conclusion}
	
%	\bibliography{bibliography}
    \printbibliography[heading=bibintoc]
    \begin{appendix}
        \section{LEF Solver}
Pour aider le développement et les tests des outils d'analyse, une application avec GUI a été développée grâce à \texttt{PyQt5} (\Cref{fig-lefsolver}). Elle permet le calcul de toutes les mesures présentées en~\Cref{sec-analyse} et de visualiser les différentes solutions trouvées par ASP. Des fonctionnalités d'exportations sont également présentes, très utiles pour l'intégration de nouveaux niveaux dans l'application pour ou pour la partie apprentissage.

\begin{figure}[ht!]
\centering
\includegraphics[width=0.7\linewidth]{lefsolver}
\caption{Capture d'écran du LEF solver}
\label{fig-lefsolver}
\end{figure}
        
        \section{Diagramme de classe d'Equity}
\begin{figure}[ht!]
\centering
\begin{tikzpicture}
    \umlsimpleclass{GameApplication}
    \umlsimpleclass[x=4.5]{LevelLoader}
    \umluniassoc{GameApplication}{LevelLoader}
    \begin{umlpackage}[x=8]{Models}
        \umlsimpleclass{Grid}
        \umlsimpleclass[y=-2]{Model}
        \umlassoc[mult2=1, mult1=1]{Model}{Grid}
        \umlsimpleclass[y=-4]{Level}
        \umlassoc[mult2=1, mult1=1]{Model}{Level}
        \umlsimpleinterface[x=3,y=-2]{IPiece}
        \umlassoc[mult1=1,mult2=*]{Model}{IPiece}
        \umlsimpleclass[x=2.5, y=-4]{Actor}
        \umlsimpleclass[x=5, y=-4]{Preference}
        \umlimpl{Actor}{IPiece}
        \umlimpl{Preference}{IPiece}
        \umlsimpleclass[x=4, y=-6.5]{Position}
        \umlassoc[mult1=1, mult2=1]{Actor}{Position}
        \umlassoc[mult1=1, mult2=1]{Preference}{Position}
    \end{umlpackage}
    \umluniassoc[geometry=-|-]{LevelLoader}{Level}
    \begin{umlpackage}[fill=red!20, y=-2]{Activities}
        \umlsimpleclass{MainMenuActivity}
        \umlsimpleclass[y=-1.5]{LevelMenuActivity}
        \umlassoc{MainMenuActivity}{LevelMenuActivity}
    \end{umlpackage}
    \umluniassoc[anchor2=50]{GameApplication}{Activities}
    \begin{umlpackage}[fill=red!20, y=-6]{Views}
        \umlsimpleclass{RecyclerView}
    \end{umlpackage}
    \umlassoc[mult1=1, mult2=*, anchors=-35 and 35]{LevelMenuActivity}{RecyclerView}
    \umlassoc[geometry=-|-]{RecyclerView}{Level}
    \begin{umlpackage}[y=-9]{GoogleSheets}
        \umlsimpleclass{GoogleSheetsWriteUtil}
        \umlsimpleclass[y=-1.5]{SheetsServiceUtil}
        \umluniassoc{GoogleSheetsWriteUtil}{SheetsServiceUtil}
        \umlsimpleclass[x=6, type=abstract]{AsyncTask}
        \umluniassoc{GoogleSheetsWriteUtil}{AsyncTask}
        \umlsimpleclass[y=-2, x=4]{WriteUserInfo}
        \umlimpl{WriteUserInfo}{AsyncTask}
        \umlsimpleclass[y=-2, x=8]{WriteUserEvaluation}
        \umlimpl{WriteUserEvaluation}{AsyncTask}
        \umlsimpleclass[x=10]{ModifyUserProfile}
        \umlimpl{ModifyUserProfile}{AsyncTask}
    \end{umlpackage}
    \umluniassoc[geometry=-|, anchor2=70]{Activities}{GoogleSheets}
\end{tikzpicture}
\caption{Diagramme de classe non exhaustif}
\end{figure}
\end{appendix}
\end{document}